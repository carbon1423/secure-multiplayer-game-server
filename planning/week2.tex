\documentclass{article}
\usepackage[margin=1in]{geometry}
\usepackage{enumitem}
\title{Week 2 Checklist: Handle Multiple Clients}
\date{}
\begin{document}
\maketitle

\section*{Goal}
Extend the server to support multiple clients. Use threading or I/O multiplexing to handle multiple simultaneous connections. Update the README and test the new functionality.

\section*{1. Threading or Multiplexing}
\begin{itemize}[label=--]
    \item Choose a strategy: \texttt{pthread} (threads) or \texttt{select()} (I/O multiplexing)
    \item If using threads:
    \begin{itemize}
        \item Create a new thread for each client using \texttt{pthread\_create()}
        \item Detach or join threads appropriately
    \end{itemize}
    \item If using \texttt{select()}:
    \begin{itemize}
        \item Add listening socket and client sockets to fd set
        \item Use \texttt{FD\_SET}, \texttt{FD\_ISSET}, and \texttt{select()} loop
    \end{itemize}
\end{itemize}

\section*{2. Server Logic Changes}
\begin{itemize}[label=--]
    \item Accept multiple client connections in a loop
    \item Ensure each client gets independent handling
    \item Add printing/logging for new connections
    \item Ensure clean handling of client disconnects
\end{itemize}

\section*{3. Testing}
\begin{itemize}[label=--]
    \item Run multiple \texttt{telnet} or \texttt{nc} sessions simultaneously
    \item Confirm each client can send and receive messages independently
    \item Ensure server does not crash or hang with more than one client
    \item Validate graceful handling of one client disconnecting while others remain
\end{itemize}

\section*{4. Documentation}
\begin{itemize}[label=--]
    \item Update \texttt{README.md}:
    \begin{itemize}
        \item Note new multi-client support
        \item Explain which strategy was used (\texttt{pthread} or \texttt{select})
        \item Update testing instructions
    \end{itemize}
    \item Commit code and README updates
\end{itemize}

\section*{5. Done}
\begin{itemize}[label=--]
    \item Push updates to GitHub
    \item Mark Week 2 complete in project plan
    \item Note any bugs or limitations to fix in Week 3
\end{itemize}

\end{document}
